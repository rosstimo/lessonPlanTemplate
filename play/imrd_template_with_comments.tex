
\documentclass{article}
\usepackage[utf8]{inputenc}
\usepackage{lipsum}  % This package generates dummy text; remove it in your actual document

\title{Your Research Title Here}
\author{Author Name}
\date{\today}

\begin{document}

\maketitle

\begin{abstract}
\textbf{Introduction:} \lipsum[1] % Replace with your actual introduction (25% of the abstract)
\textbf{Methods:} \lipsum[2] % Replace with your actual methods (25% of the abstract)
\textbf{Results:} \lipsum[3] % Replace with your actual results (35% of the abstract)
\textbf{Discussion:} \lipsum[4] % Replace with your actual discussion (15% of the abstract)
\end{abstract}

\section{Introduction}
% Introduction)
• 25% (
\lipsum[1] % Replace with your introduction content

\section{Methods}
% Methods)
• 35% (
\lipsum[2] % Replace with your methods content

\section{Results}
% Results
A.
Table	3	shows	that	Spam	Filter	A	correctly	filtered	more	junk	emails	than	Filter	B.1	Filter	A	correctly	filtered	88%	of	
junk	emails	whereas	filter	B	only	filtered	63%	correctly.2 However, Filter A takes longer to run than Filter B.4 This 
increased run time is due to the type of programming language used in Filter A.5	These	findings	overall	suggest	that	
Spam	Filter	A	is	a	better	filter	than	Filter	B	even	though	it	takes	longer	to	run.8
B.
Fig.	3	shows	that	the	electrical	conductivity	of	the	Cu-doped	ZnO	is	much	lower	than	that	of	the	undoped	ZnO.1 The 
electrical	conductivity	of	even	the	100	ppm	Cu-doped	ZnO	specimen	was	about	3	orders	of	magnitude	lower	than	
that	of	the	undoped	ZnO.2 As the doped Cu content increased, the electrical conductivity gradually decreased.3 As 
a	result,	the	1000	ppm	Cu-doped	ZnO	had	the	electrical	conductivity	5	orders	of	magnitude	lower	than	that	of	the	
undoped	ZnO.8
\lipsum[3] % Replace with your results content

\section{Discussion}
% Discussion
The data collected from this small study suggests that verbal instructions are not needed to 
complete a simple assembly task and may even interfere with the task
% The participants who 
received words plus pictures made more errors, took longer to complete the task, and were less 
confident	that	they	had	completed	the	task	correctly	than	participants	who	received	pictures	
alone.	One	reason	for	this	finding	may	be	the	simplicity	of	the	task	since	none	of	the	guidelines	
we examined suggest that textual information would interfere with visual instructions.
Our study is hampered by the small number and homogeneity of our participants
% All of our 
participants	were	college	students	and	this	may	have	affected	our	results.	Additional	research	
might	examine	whether	older	participants	would	benefit	from	verbal	instructions	accompanying	
pictures.		More	research	is	also	needed	examining	different	tasks.	Our	study	involved	a	highly	
physical task (constructing a lego vehicle)
% Future research should examine how pictures and 
verbal instructions might interact on a more conceptual task, such as installing and using a 
software program.
Based on this limited analysis, we recommend that instruction writers consider excluding verbal 
instructions on a simple assembly task
%  Our results indicate that verbal instructions may in 
some cases interfere with users’ abilities to follow pictorial directions.
Summarize results
Explain results
Flaws
Future research
Implications
\lipsum[4] % Replace with your discussion content

\end{document}
