\documentclass[conference]{IEEEtran}

\usepackage{graphicx}
\usepackage{amsmath}
\usepackage[backend=biber, style=ieee]{biblatex}
\addbibresource{references.bib}

\begin{document}

\title{Your Paper Title}

\author{\IEEEauthorblockN{Author One}
\IEEEauthorblockA{Department\\
University\\
City, Country\\
Email: author.one@email.com}
\and
\IEEEauthorblockN{Author Two}
\IEEEauthorblockA{Department\\
University\\
City, Country\\
Email: author.two@email.com}}

\maketitle

\begin{abstract}
Your abstract here.
\end{abstract}

\begin{IEEEkeywords}
Keyword1, Keyword2, Keyword3
\end{IEEEkeywords}

\section{Introduction}
PID controllers are widely used in various engineering applications. One of the challenges in using PID controllers is tuning them for specific systems like BLDC motors. A study by Navaneethakkannan and Sudha~\cite{navaneethakkannan2013} provides a simple method for tuning PID controllers specifically for BLDC motors. Further advancements in PID controller design have also been explored, such as the use of fuzzy logic, as discussed by Yeung~\cite{yeung2014}. Wang~\cite{wang2012} provides a new design strategy for PID controllers, offering a different perspective on their optimization.

\section{Literature Review}
Your literature review content.

\section{Methods}
Your methods content.

\section{Results}
Your results content.

\section{Discussion}
Your discussion content.

\section{Conclusion}
Your conclusion content.

\section*{Acknowledgment}
Your acknowledgment content.

\printbibliography

\end{document}

