% see: https://texample.net/tikz/examples/tikz-timing/
% see: https://ctan.org/pkg/tikz-timing

\documentclass{article}
\usepackage{tikz-timing}
\usepackage{graphicx}

\begin{document}

\begin{figure}[htbp]
    \centering
    \resizebox{0.8\textwidth}{!}{
        \begin{tikztimingtable}
            % Define signal A with three cycles
            A & 2c3{2{4c}} \\
            % Define signal B with three cycles, starting 90 degrees (one quarter period) after A
            B & 3{2{4c}}2c \\
            \extracode
            \begin{pgfonlayer}{background}
                % Draw and label the phase relationship
                % \draw[<->, thick] (1,-0.5) -- node[above] {$\theta$} (2,-0.5);
                \draw[<->, thick] (1,-0.5) -- node[pos=0.5, above, yshift=0.30cm] {$\theta$} (2,-0.5);
                % Draw and label the period of A
                \draw[<->, thick] (1,3) -- node[above] {$T$} (5,3);
                % Add vertical dashed lines at the first and second rising edges of A
                % \vertlines[help lines, dashed]{1 ,2 ,5}
                \draw[dashed, help lines] (1,-2) -- (1,3);
                \draw[dashed, help lines] (2,-2) -- (2,0);
                \draw[dashed, help lines] (5,-2) -- (5,3);
            \end{pgfonlayer}
        \end{tikztimingtable}
    }
    \caption{Timing diagram showing three cycles of signals A and B with phase relationship and period of A labeled.}
    \label{fig:timing-diagram}
\end{figure}

\end{document}

