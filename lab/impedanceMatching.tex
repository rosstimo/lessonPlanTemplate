\documentclass[main.tex]{subfiles}
%%------ Preamble specific to this subfile only. This will be used when this document is compiled on its own. When the main document compiles this preabble is ignored. 
%%-
\begin{document}
\section{Impedance Matching Considerations}
\label{sec:ImpedanceSection}

\subsubsection*{Objectives}
\begin{enumerate}
    \item Describe in writing the effects of varying \zout\ vs. \RL\ ratios.
    \item Demonstrate the ability to accurately represent AC load line(s) with varying \zout\ vs. \RL\ ratios.
    \item Demonstrate the ability to accurately analyze, predict and measure the effects of varying \zout\ vs. \RL\ ratios.
    \item Define in writing key terms including but not limited to: Impedance Matched, Swamping, \voutmax, \vinmax.
\end{enumerate}

\subsubsection*{Figures/Tables}
\begin{enumerate}
  \item Refer to Figure 3-1-1 Universal Biased Common Emitter Amplifier.
  \item Include the schematic figure, tables and load line together for ease of review and comparison.
  \item The AC/DC load line will be the primary representation of circuit operation for this section.
\end{enumerate}

% \subsubsection*{Table 4-1-2: Changing Load Calculations}

% \begin{center}
  \begin{table}[h]
    \caption{Changing Load Calculations}
    % \begin{tabular}{|c|c|c|c|c|c|c|c|c|c|}
    % \begin{tabular}{|m{1cm}|m{1cm}|m{1cm}|m{1cm}|m{1cm}|m{1cm}|m{1cm}|m{1cm}|m{1cm}|m{1cm}|}
    \begin{tabularx}{\textwidth}{|X|X|X|X|X|X|X|X|X|X|X|}
      \hline
      \RL & \re & \zin & \zout & \Av & \Ai & \Ap & \vinmax & \voutmax & \vcecut & \icsat  \\
      \hline
      1.6\kohm & & & & & & & & & & \\
      \hline
      3.3\kohm & & & & & & & & & & \\
      \hline
      6.8\kohm & & & & & & & & & & \\
      \hline
    \end{tabularx}
  \end{table}

   \begin{table}[h]
    \caption{Changing Load Calculations}
    % \begin{tabular}{|c|c|c|c|c|c|c|c|c|c|}
    % \begin{tabular}{|m{1cm}|m{1cm}|m{1cm}|m{1cm}|m{1cm}|m{1cm}|m{1cm}|m{1cm}|m{1cm}|m{1cm}|}
    \begin{tabularx}{0.6\textwidth}{|X|X|X|X|X|}
      \hline
      \RL & \Av & \Ai & \Ap & \voutmax \\
      \hline
      1.6\kohm & & & &  \\
      \hline
      3.3\kohm & & & &  \\
      \hline
      6.8\kohm & & & &  \\
      \hline
    \end{tabularx}
  \end{table}
  % \end{center}

\newpage

\subsubsection*{Theory of Operation}
\begin{enumerate}
  \item Review the derived amplifier formulas from previous sections
  \item Determine what formulas contain \zout\ and \RL\
  \item Proper \Av\ measurements will verify predictions and assumptions
\end{enumerate}

\subsubsection*{Procedure}
\begin{enumerate}
  \item kjhkjh  
\end{enumerate}

\subsubsection*{Calculations}
\begin{enumerate}
    Step by step (check list)
Calculate Av, Ai, Ap and voutmax when RL = 3.3kΩ
Calculate Av, Ai, Ap and voutmax when RL = 6.8kΩ
Calculate Av, Ai, Ap and voutmax when RL = 1.6kΩ
Use Av and voutmax to calculate vinmax for each load value
\end{enumerate}

\subsubsection*{Measurements}
\begin{enumerate}
For each of the three RL values from above:
Measure Av with vgen set to half of calculated vinmax or lower and frequency set to 10kHz
Use the same generator settings for all Av measurements
Measure/verify voutmax 
\end{enumerate}

\subsubsection*{Waveforms/Diagrams}
\begin{enumerate}
Draw a properly scaled predicted DC and AC load line (big and readable)
Include DC and all three AC load lines on the same figure (labeled and color coded)
Include properly aligned and scaled Av measurement waveform for each RL value
\end{enumerate}

\subsubsection*{Conclusion}
\begin{enumerate}
  \item kjhkjh  
\end{enumerate}

\char{0135}
\end{document}
