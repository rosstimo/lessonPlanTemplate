\documentclass[main.tex]{subfiles}
%%------ Preamble specific to this subfile only. This will be used when this document is compiled on its own. When the main document compiles this preabble is ignored. 

%%---------------------------------------------------------------------------%%

\begin{document}
\section*{Student Assessment} 
\setlength{\parindent}{0in}

In the context of a lesson plan, the "Assessment" section refers to how the instructor plans to evaluate or measure students' understanding and learning of the lesson's objectives.
% Assessment can take various forms, including but not limited to:
% 1. **Quizzes and Tests:** These can be given during or after the lesson to evaluate students' knowledge and understanding.
% 2. **Classroom Activities:** Such as group work, presentations, and in-class assignments. These can provide an opportunity to assess both individual and group understanding.
% 3. **Homework Assignments:** Work assigned to be completed outside of class can provide another way to assess student understanding.
% 4. **Class Participation:** The level and quality of student involvement in class discussions can also be a part of assessment.
% 5. **Observation:** Teachers can informally assess student understanding by observing their engagement, asking questions, and encouraging classroom discussion.
% In your lesson plan, under the "Assessment" section, you would list the methods you intend to use to gauge students' understanding of the lesson's material. For example, you might plan to give a short quiz at the end of the lesson, assign homework, or observe student participation during a class discussion.
\begin{itemize}

  \item \textbf{Observation:}\\
    Teachers can informally assess student understanding by observing their engagement, asking questions, and encouraging classroom discussion.
    \begin{itemize}
      \item True or False: \LaTeX{} is a great tool for creating lesson plans? A:True 
      \item What is 2 + 2? A:4
      \item Review topics as needed.
      \item Take specific lesson plan notes to improve delivery
      \item TODO Include writing prompt questions here
    \end{itemize}

  \item \textbf{Class Participation:}\\
    The level and quality of student involvement in class discussions can also be a part of assessment.
    \begin{itemize}
      \item Use the Professionalism rubric as a reference. Take specific notes  
      \item TODO Include writing prompt questions here
    \end{itemize}

  \item \textbf{Classroom Activities:}\\ 
    Such as group work, presentations, and in-class assignments. These can provide an opportunity to assess both individual and group understanding.
    \begin{itemize}
      \item Small white board Q \& A 
      \item Group work on the whiteboard
      \item jeapordy game
    \end{itemize}

  \item \textbf{Quizzes and Tests:}\\
    These can be given during or after the lesson to evaluate students' knowledge and understanding.
    \begin{itemize}
      \item Warm Up Quiz Comeing Soon!
      \item This will be on next weeks Friday test.
    \end{itemize}

  \item \textbf{Homework Assignments:}\\
    Work assigned to be completed outside of class can provide another way to assess student understanding.
    \begin{itemize}
      \item Complete Moodle quiz before next class session 
    \end{itemize}

\end{itemize}

\end{document}
