%% Thin is an attempt at a generic preamble that can be called instead of copies 
%% a bunch of times. Any specific/minor document tweaks can be added to the main document
%% preamble and should therefore override or add to these settings.
%%---------------------------------------------------------------------------%%

%%------   page format
% set text encoding to utf-8
\usepackage[utf8]{inputenc}

% page margins
\usepackage[top=0.5in, bottom=0.5in, left=1in, right=1in]{geometry}

% contrl end of line wrap/hyphenation 
\usepackage{ragged2e}

% More granular control over title format of part, section, subsection, paragragh ...
\usepackage{titlesec}%custom section formatting
\usepackage{titling}% custom title formatting
% Custom section spacing and formatting
% \titleformat{\part}{\Huge\scshape\filcenter}{}{1em}{}
% \titleformat{\section}{\Large\bf\raggedright}{}{1em}{}[{\titlerule[2pt]}]
% \titlespacing{\section}{0pt}{3pt}{7pt}
% \titleformat{\subsection}{\large\bfseries\centering}{}{0em}{\underline}[\rule{3cm}{.2pt}]
% \titlespacing{\subsection}{0pt}{7pt}{7pt}

% list formatting
% to use the three basic list in line: just add the package option inline and then the 
% environments enumerate*, itemize* and description*.
\usepackage{enumitem} 

%%------ LaTeX mechanics
\usepackage{subfiles} % allows compile of only subfile
\usepackage{xifthen}

%%------   debug/development
\usepackage{blindtext} % Dummy text and tools 
%\Blindtext [2] [1] % two paragraphs 1 textblock

\usepackage{comment} % adds multiline comments \begin{comment} ... \end{comment}

%%------   math tools
\usepackage{mathtools} % new, contains amsmath
\usepackage{upgreek} % better looking Greek

%%------   image/figure tools
\usepackage{graphicx}
\graphicspath{ {./images/} } %   implies images dir in current dir
\usepackage{wrapfig}
\usepackage[font=scriptsize,labelfont=bf]{caption}

%%------ figure/diagram tools

%%------ table tools

%%------
