\documentclass[main.tex]{subfiles}
%%------ Preamble specific to this subfile only. This will be used when this document is compiled on its own. When the main document compiles this preabble is ignored. 
%%---------------------------------------------------------------------------%%
\begin{document}
\section*{Objectives}
\begin{comment}

    A well-written objective, often referred to as a learning objective or learning outcome, should be:
    1. **Specific:** It should clearly define what a student will be able to do by the end of the lesson.
    2. **Measurable:** It should be quantifiable in some way, whether that's through a test, an activity, or another form of assessment.
    3. **Achievable:** It should be realistic considering the students' current level of knowledge and the time allotted.
    4. **Relevant:** It should align with the overall goals of the course or curriculum.
    5. **Time-bound:** It should be achievable within a certain time frame, usually by the end of the lesson, unit, or course.
    This framework is often referred to as "SMART" objectives.
    The assessment should then be designed to directly measure whether or not each objective has been met. For instance, if your objective is "By the end of this lesson, students will be able to identify and explain the five major causes of World War I", your assessment might be a quiz that requires students to do exactly that. 

    Moreover, there should be a clear alignment (often referred to as "alignment of assessment") between your learning objectives and your assessment methods. For example, if one of your objectives requires critical thinking, then your assessment should not simply require factual recall. 

    Additionally, using a variety of assessment methods (such as quizzes, homework assignments, in-class discussions, and so on) can provide a more comprehensive picture of whether the students have met the objectives, as different students may excel in different types of assessments. 

    Ultimately, the objectives should guide the instruction and the assessment should measure the degree to which the instruction has enabled students to meet the objectives. This linkage between the objectives and assessment is critical to ensuring that the lesson is effective and that students' learning is maximized.

\end{comment}
\begin{enumerate}
  \item  Fill in objective here
  \item  Fill in objective here
\end{enumerate}
\end{document}

